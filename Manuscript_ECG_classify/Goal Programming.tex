\documentclass[aps,pra,twocolumn,showpacs,superscriptaddress,groupedaddress]{revtex4}  % for review and submission
%\documentclass[aps,preprint,showpacs,superscriptaddress,groupedaddress]{revtex4}  % for double-spaced preprint
\usepackage{graphicx}  % needed for figures
\usepackage{dcolumn}   % needed for some tables
\usepackage{bm}        % for math
\usepackage{amssymb}

\hyphenation{ALPGEN}
\hyphenation{EVTGEN}
\hyphenation{PYTHIA}


\newcommand{\Del}{\nabla}
\newcommand{\ex}[1]{\langle #1 \rangle}
\newcommand{\exx}[1]{\langle \langle #1 \rangle\rangle}
\newcommand{\rABgoes}{\stackrel{{\textstyle \longrightarrow}}{{\scriptstyle
r_{AB} \rightarrow \infty}}}
\newcommand{\ggoes}{\stackrel{{\textstyle \longrightarrow}}{{\scriptstyle g
\rightarrow \infty}}}
\newcommand{\arccosh}{\text{ arccosh}}

\newcommand{\ket}[1]{| #1 \rangle}
\newcommand{\bra}[1]{\langle #1 |}
\newcommand{\rb}[1]{\left( #1 \right)}
\newcommand{\mb}[1]{\mathbf{#1}}
\newcommand{\ew}[1]{\langle #1 \rangle}
\newcommand{\beq}{\begin{eqnarray}}
\newcommand{\eeq}{\end{eqnarray}}
\newcommand{\blg}[1]{\mathrm{lg}\left[  #1right]}
\newcommand{\svec}{\mbox{\boldmath$\sigma$}}
\newcommand{\op}[2]{| #1 \rangle \langle #2 |}
\newcommand{\sech}{\mathrm{sech}}
\newcommand{\sfrac}[2]{\begin{array}{c}\frac{#1}{#2}\end{array}}
\newcommand{\cum}[1]{\langle\langle #1 \rangle \rangle}
\newcommand{\eq}[1]{Eq.~(\ref{#1})}
\newcommand{\fig}[1]{Fig.~\ref{#1}}
\newcommand{\ul}[1]{{\underline{#1}}}
\begin{document}


\title{Quantitative characterization of multiobjective via
 goal programming}

\author{Lijuan Zheng}
\address{
South China University of Technology, Guangzhou 510641, China}

\date{\today}


%\begin{abstract}
%
%\end{abstract}
%\pacs{}
 \maketitle



\begin{figure}
\setlength{\abovecaptionskip}{7pt}
\includegraphics[width=1\columnwidth]{fig1}
\caption{(Color online)  Density profile of the optimization function
 $\mathcal{O}$ in the goal programming model for several
 cases: (a) $\eta_1=\eta_2=1$ and $\Delta_C=\Delta_B=0.1$; (b)
 $\eta_1=\eta_2=1$ and $\Delta_C=\Delta_B=0.2$; (c) $\eta_1=1,
 \eta_2=0.5$ and $\Delta_C=\Delta_B=0.1$; (d) $\eta_1=0.5, \eta_2=1$ and
 $\Delta_C=\Delta_B=0.1$.}
\end{figure}

The results in Section IV show a clear trade-off relation
between the accuracy and the cost in ECG compression and transmission.
Let us examine this relation via a sophisticated method, goal
programming\,\cite{Goal}.
We formulate our problem setting more specifically; let $\mathcal{C}$ be the cost and $\mathcal{A}$ be the accuracy.
We overall consider the multiobjective optimization 
 such that $\mathcal{C} \le \Delta_{C}$ and $\mathcal{A} \le \Delta_{A}$ for given
(small) positive constants $\Delta_{C}$ and $\Delta_{A}$.
The two parameters $\Delta_{C}$ and $\Delta_{A}$ are regarded as,
respectively, admissible cost and permissible error.
Thus, we can obtain a good measurement scenario to decrease the cost (i.e., minimize $\mathcal{C}$) while reducing the error (i.e.,
minimizing $\mathcal{A}$).


Goal programming \cite{Goal} provides an optimization method to deal
with two (or more than two) conflicting objectives and it is widely used in
mathematics, information theory and engineering.
Instead of finding solutions which absolutely minimize or maximize
objective functions,
the task of goal programming is to find solutions that, if possible,
satisfy a set of goals, or otherwise violate the goals minimally.
This makes the approach more appealing to practical designers, compared
to other optimization methods (e.g., linear programming models).
Let us describe our measurement problem in terms of goal programming:
\begin{eqnarray}
&&
{\rm Minimize}\quad
\mathcal{O}\equiv \eta_{1}\delta_{1}^{+} + \eta_{2}\delta_{2}^{+},
 \nonumber \\
&&
{\rm subject\,\,to}\,
\left\{
\begin{array}{l}
\mathcal{C}(\vec{x})-\delta_{1}^{+}+\delta_{1}^{-} = \Delta_{C} \\
\mathcal{A}(\vec{x})-\delta_{2}^{+}+\delta_{2}^{-} = \Delta_{A} \\
\delta_{1}^{\pm},\delta_{2}^{\pm} \ge 0 
%\\
%\vec{x}=(t\Omega/2\pi, k\Omega) \in \mathcal{M}
\end{array}
\right. .
\label{GP}
\end{eqnarray}
The weight factors $\eta_{1}$ and $\eta_{2}$ are given positive
number, and represent the relative priority of objectives.
If $\eta_{2} > \eta_{1}$, the condition for the
error is prior to the one for the cost, and vice versa.
The condition for the cost ($\mathcal{C}\le \Delta_{C}$) is reformulated by
the relation
\(
\mathcal{C} +\delta_{1}^{+}-\delta_{1}^{-} = \Delta_{C}
\), with the deviations between the admissible error and the actual
value, $\delta_{1}^{+}$ and $\delta_{1}^{-}$.
When $\mathcal{C} > \Delta_{C}$ ($\mathcal{C} \le \Delta_{C}$),
we set $\delta_{1}^{+}=\mathcal{C}-\Delta_{C}$ and $\delta_{1}^{-}=0$
($\delta_{1}^{+}=0$ and $\delta_{1}^{-}=\Delta_{C}-\mathcal{C}$).
Similarly, we can set $\delta_{2}^{\pm}$ via
$\mathcal{A}-\delta_{2}^{+}+\delta_{2}^{-}=\Delta_{A}$.
The smaller $\mathcal{O}$, the better performance of the measurement scenario.
The minimum value of $\mathcal{O}$ ($\mathcal{O}=0$) corresponds
to the best solution.

Figure 1 shows the contour profile of $\mathcal{O}$ for various cases,
based on the cost $\mathcal{C}$ and overall error $\mathcal{A}$.
In both Figs.\,1(a) and (b), the confidence and the backaction
have equal importance ($\eta_1=\eta_2=1$).
In Fig.1(a) we examine the case that $\Delta_{C}=0.1$ and
$\Delta_{B}=0.1$.
We find that the best solution ($\mathcal{O}=0$) appears in the dark
blue area.
If we relax the restriction to $\Delta_C=\Delta_B=0.2$, we find that
more solutions for the optimization $\mathcal{O}=0$, as seen in Fig.\,1(b).
We also examine the cases when the cost has a different importance, or weight,
than the accuracy, as seen in Figs.\,1(c) and (d).
The solution for the case where the cost is more important than the error ($\eta_1=1, \eta_2=0.5$) is given in Fig. 1(c). The solution for the opposite situation ($\eta_1=0.5, \eta_2=1$) is given in Fig. 1(d).
In the above cases, we have considered a double-criterion goal
programming problem and we find that it is convenient for discussing
the trade-off relation between cost and error.



\begin{thebibliography}{99}

\bibitem{Goal}M. J. Schniederjans,
\emph{Goal Programming: Methodology and Applications} (Kluwer, Boston, 1995).



\end{thebibliography}

\end{document}
